\documentclass[twoside,spanish,11pt]{book}

\usepackage[utf8]{inputenc}
\usepackage[spanish,es-tabla,english]{babel}

\usepackage[usenames]{xcolor}

\usepackage{array}
\usepackage{subcaption}
\usepackage[toc,page]{appendix}
\usepackage{multirow}
\usepackage{indentfirst}
\usepackage{cite}

\usepackage{ulem}
\renewcommand{\uline}[1]{\textit{#1}}
\definecolor{MyOrange}{rgb}{0.9,0.6,0.0}
\definecolor{MyDarkBlue}{rgb}{0,0,0.8}
\definecolor{MyDarkRed}{rgb}{0.6,0,0.0}
\newcommand{\borrar}[1]{\textcolor{MyDarkRed}{\sout{#1}}} %\sout}
\newcommand{\nuevo}[1]{\textcolor{MyDarkBlue}{\uline{#1}}} %\sout}
\newcommand{\nota}[1]{\textcolor{MyOrange}{#1}} %\sout}

\reversemarginpar
\usepackage{setspace}
\usepackage{todonotes}
\newcounter{mycomment}
\newcommand{\mycomment}[3][]{%
% initials of the author (optional) + note in the margin
    \refstepcounter{mycomment}%
    {%
        \setstretch{0.7}% spacing
        \todo[color=#3,size=\scriptsize]{%
            \tiny{\textbf{(\themycomment)[\uppercase{#1}]:}~#2}}%
    }}
% For notes, corrections and suggestions
\definecolor{GRCOLOR}{rgb}{1,0.2,0.5}
\definecolor{ODRCOLOR}{rgb}{0.2,0.75,0.25}
\definecolor{LRCOLOR}{rgb}{0.2,0.3,1}
\definecolor{MyDarkRed}{rgb}{0.6,0,0.0}

\newcommand{\GR}[1] { \mycomment[PT]{#1}{GRCOLOR}}
\newcommand{\ODR}[1]{ \mycomment[ODR]{#1}{ODRCOLOR}}
\newcommand{\LR}[1]{ \mycomment[SM]{#1}{LRCOLOR}}

\newcommand{\newGR}[1]{\textcolor{PTCOLOR}{\uline{#1}}} %\sout}
\newcommand{\delGR}[1]{\textcolor{PTCOLOR}{\sout{#1}}} %\sout}
\newcommand{\newODR}[1]{\textcolor{ODRCOLOR}{\uline{#1}}} %\sout}
\newcommand{\delODR}[1]{\textcolor{ODRCOLOR}{\sout{#1}}} %\sout}
\newcommand{\newLR}[1]{\textcolor{SMCOLOR}{\uline{#1}}} %\sout}
\newcommand{\delLR}[1]{\textcolor{SMCOLOR}{\sout{#1}}} %\sout}
\newcommand{\note}[1]{\textcolor{MyDarkRed}{#1}} %\sout}
\newcommand{\deletenote}[1]{\textcolor{MyDarkRed}{\sout{#1}}} %\sout}


%\newcommand{\delsg}[1]{\textcolor{SGCOLOR}{\sout{#1}}} %\sout}

\usepackage{graphicx}
\usepackage{listings}
\usepackage{float}

\usepackage{rotating}

\usepackage[hyphens]{url} % Hay que cargarlo antes que hyperref para URLs largas
\usepackage[hidelinks=true]{hyperref}
\hypersetup{
%	colorlinks,
    linkcolor={blue!80!black},
    citecolor={blue!80!black},
    urlcolor={blue!80!black}
}
%\hypersetup{
%    colorlinks,
%    linkcolor={black},
%    citecolor={black},
%    urlcolor={black}
%}

\usepackage{marginnote}
\renewcommand*{\marginfont}{\scriptsize\color{red}\sffamily}

\usepackage{enumitem}
\setitemize{noitemsep,topsep=0pt,parsep=0pt,partopsep=0pt,itemsep=2mm}%
\setenumerate{noitemsep,topsep=0pt,parsep=0pt,partopsep=0pt,itemsep=2mm}
% \setlist{noitemsep}
%% \setlist[itemize]{itemsep=0.5mm}
%% \setlist[enumerate]{itemsep=0.5mm}

\usepackage[textfont={footnotesize,sf},labelfont={footnotesize,sf,bf}]{caption}

\renewcommand{\labelitemi}{$\bullet$}
\renewcommand{\labelitemii}{$\circ$}
\renewcommand{\labelitemiii}{--}

\setlength{\parskip}{3mm}

%% \usepackage{natbib}
%% \usepackage{chapterbib}

\renewcommand{\baselinestretch}{1.05}


\setcounter{tocdepth}{2}
\setcounter{secnumdepth}{5}

\usepackage[big,sf,bf,pagestyles]{titlesec}

\usepackage{xspace}

%% \usepackage[a4paper,
%%   total={145mm,237mm},
%%   left=40mm,
%%   top=25mm,
%%   marginparwidth=2.0cm]{geometry}

\usepackage[a4paper,
    total={155mm,237mm},
    left=35mm,
    top=32mm,
    marginparwidth=3.0cm,
]{geometry}

\lstdefinestyle{java} {
    language=Java,
    morekeywords={record, var}
}

\usepackage{multirow}

\usepackage{fancyhdr}
\pagestyle{fancy}

\fancyhead[LE]{}
\fancyhead[RO]{}
\fancyhead[LO]{\sf{\footnotesize\leftmark}}
\fancyhead[RE]{\sf{\footnotesize\rightmark}}

%\fancyhead[LE,RO]{\scriptsize\textbf\elautor}
%\fancyhead[RE,LO]{\scriptsize\textbf\leftmark}

%\renewcommand{\chaptermark}[1]{
%	\markboth{\chaptername
%		\ \thechapter.\ #1}{}}

\renewcommand{\sectionmark}[1]{\markright{\thesection.\ #1}}


\setlength{\headwidth}{\textwidth}

\usepackage{lmodern,textcomp}

\title{Trabajo de Fin de Grado}
\begin{document}

    \include{chapters/front}

    \pagestyle{empty}

    \pagenumbering{gobble}

    \frontmatter

    \pagestyle{plain}

    \chapter{Resumen} \label{ch:resumen}

Este Trabajo Fin de Grado contempla el diseño y desarrollo de \textit{JAMS},
un entorno de desarrollo integrado moderno, ligero y altamente modificable
para proyectos en lenguaje ensamblador.
El entorno de desarrollo incorpora por defecto un editor, ensamblador y si-mulador
para la arquitectura \textit{MIPS32}, con un enfoque educativo.
Para ello, la aplicación presenta varias herramientas que permiten
visualizar el estado del proyecto desde diferentes puntos de vista.\\
Con esto, se pretende contribuir a una enseñanza de mayor calidad para
los alumnos que cursan alguna asignatura relacionada con arquitectura de computadores,
procesadores o lenguajes de programación de bajo nivel.

    \clearpage{\pagestyle{empty}\cleardoublepage}

    \selectlanguage{english}

    \chapter{Abstract} \label{ch:abstract}

This Bachelor's Thesis to obtain the Degree of Coputing Engineering contemplates the design and development of \textit{JAMS},
a modern, lightweight and highly modifiable integrated
development environment for assembly language projects.
The development environment incorporates by default an editor,
assembler and simulator for the MIPS32 architecture, with an educational approach.
For this purpose, the application presents several tools that allow visualizing
the project status from different points of view.
With this, it is intended to contribute to a better quality
teaching for students taking a subject related to computer architecture,
processors or low-level programming languages.

    \clearpage{\pagestyle{empty}\cleardoublepage}

    \selectlanguage{spanish}


%	\listoftodos

    \setcounter{tocdepth}{3}
    \tableofcontents
    \clearpage{\pagestyle{empty}\cleardoublepage}

    \listoffigures
    \clearpage{\pagestyle{empty}\cleardoublepage}

    \mainmatter

    \pagestyle{fancy}

    \clearpage{\pagestyle{empty}\cleardoublepage}

    \include{chapters/1-introduction}
    \clearpage{\pagestyle{empty}\cleardoublepage}

    \include{chapters/2-Antecedentes}
    \clearpage{\pagestyle{empty}\cleardoublepage}

    \include{chapters/3-TecnologiesAndProyect}
    \clearpage{\pagestyle{empty}\cleardoublepage}

    \include{chapters/4-BaseDevelopment}
    \clearpage{\pagestyle{empty}\cleardoublepage}

    \include{chapters/5-EntornoMIPS32}
    \clearpage{\pagestyle{empty}\cleardoublepage}

    \chapter{Resultados}\label{ch:resultados}


\section{Resultados relativos al objetivo 1}\label{sec:resultados-relativos-al-objetivo-1}

Los resultados de este objetivo corresponden con la selección de
tecnologías adecuadas para el desarrollo de la aplicación,
eligiendo un lenguaje de programación y un \textit{framework} de
desarrollo de aplicaciones de escritorio que permitan crear
aplicaciones modernas, multiplataforma, de gran calidad,
y que permitan cargar código externo a voluntad.

El proceso de búsqueda y selección del lenguaje de programación y
del framework culminó con la elección de \textit{Java 17}
y \textit{JavaFX} respectivamente.
Con estas tecnologías se ha conseguido desarrollar una aplicación
moderna y multiplataforma, con soporte para componentes y totalmente
personalizable.

Gracias al uso de una versión de \textit{Java} moderna,
el desarrollo de la aplicación ha sido \textbf{muy rápido} y el resultado
ha sido \textbf{muy profesional}, pudiendo implementar diseños y
algoritmos rápidamente y con pocos errores.
\textit{JavaFX} ha permitido desarrollar una interfaz de usuario
que se separa del conocido y desfasado formato de las aplicaciones
\textit{Swing}.
Gracias a su fácil desarrollo y su capacidad de usar código
\textit{CSS} para definir el estilo de la aplicación, \textit{JAMS}
cuenta con un aspecto moderno y profesional.

Una de las ventajas que ha aparecido durante el
desarrollo gracias a utilizar \textit{Java} ha sido la
capacidad de poder usar otros lenguajes de programación capaces
de compilar a la \textit{JVM} para el desarrollo de componentes.
Así, el desarrollador podrá escoger entre un \textbf{amplio abanico de lenguajes}
de programación para crear su componente.
Algunos ejemplos de lenguajes de programación que compilan a la \textit{JVM}
son \textit{Scala}, \textit{Kotlin} o \textit{Groovy}.
Un ejemplo de componente desarrollado en \textit{Kotlin} sería
\textit{NES4JAMS}, que forma parte del Trabajo de Fin de Grado
realizado por el autor para la obtención del título de Grado en
Diseño y Desarrollo de Videojuegos.

\section{Resultados relativos al objetivo 2}\label{sec:resultados-relativos-al-objetivo-2}

Los resultados de este objetivo corresponden con la creación de
un entorno base y un \textit{framework} que permita \textbf{implementar
diferentes entornos y herramientas}, creando así una capa
de abstracción que ayuda a los desarrolladores y al propio
\textit{JAMS} a crear herramientas de manera rápida y sencilla.

Se considera que este objetivo se ha conseguido.
Gracias a las diferentes tecnologías que se han desarrollado
para la base, es posible crear nuevas herramientas en cuestión
de minutos.

Los requisitos establecidos en el Trabajo Fin de Grado del Grado en Diseño
y Desarrollo de Videojuegos de la misma autoría y que se apoya en el presente
trabajo han motivado la necesidad de introducir cambios y mejoras en el código
que han contribuido a que \textit{JAMS} actualmente se haya convertido en una
aplicación base robusta, con capacidad de personalizar
una gran cantidad de aspectos del entorno.

La base también permite a los usuarios más comunes e
inexpertos personalizar el entorno de desarrollo gracias a la
\textbf{extensa configuración} y la capacidad de poder producir
\textbf{paquetes de idiomas y de temas}.
El resultado de estas personalizaciones se puede apreciar
perfectamente en la figura \ref{fig:jams-collage}.

\begin{figure}[h]
    \centering
    \includegraphics[width=0.8\textwidth]{images/result/jams-collage}
    \caption{Diferentes perfiles de personalización de \textit{JAMS}}
    \label{fig:jams-collage}
\end{figure}

Por último, destacar el resultado de la \textit{interfaz de usuario}.
El uso de nodos como elemento central de la interfaz puede considerarse
un \textbf{acierto}: son componentes muy fáciles de usar y altamente personalizables.
Estos nodos son independientes entre sí, y muchos de ellos son \textbf{independientes}
de la tecnología empleada por el usuario, como es el caso del \textbf{explorador}.
Esto permite que sean herramientas \textbf{altamente reutilizables}, estando en todo
momento a disposición del desarrollador de componentes.


\section{Resultados relativos al objetivo 3}\label{sec:resultados-relativos-al-objetivo-3}

Los resultados de este objetivo corresponden con la implementación
de un entorno de desarrollo para la arquitectura \textit{MIPS32}
usando la base creada en el objetivo anterior.
El objetivo requiere de la creación de un editor, un ensamblador
y un simulador, además de diferentes herramientas que complementen
a estos tres elementos principales.

Se considera que este objetivo se ha superado
de manera óptima.
\textit{JAMS} presenta de inicio un entorno de desarrollo completo
para la arquitectura \textit{MIPS32}.

El \textbf{editor de texto} está al nivel de los editores de texto
inteligentes que se pueden encontrar en los entornos de desarrollo
actuales, \textbf{ayudando al usuario} en la mayoría de las tareas
relacionadas con la programación de una aplicación en ensamblador, como se observa
en la figura \ref{fig:mips-editor}.

\begin{figure}[ht]
    \centering
    \includegraphics[width=\textwidth]{images/result/mips-editor}
    \caption{Editor de texto proporcionando ayuda al usuario}
    \label{fig:mips-editor}
\end{figure}

El \textbf{ensamblador} es altamente \textbf{personalizable},
permitiendo que otros componentes puedan aportar nuevas instrucciones y
directivas de manera sencilla.
Este ensamblador también incorpora varias \textbf{características avanzadas}
que el usuario puede emplear, como son las \textbf{macros} y las
\textbf{etiquetas relativas}.

El \textbf{simulador} permite ejecutar código ensamblador
\textbf{MIPS32} en diferentes arquitecturas, siendo la arquitectura
uniciclo la más rápida de todas ellas, llegando a superar los
\textbf{40 millones de ciclos cada segundo}.
El simulador está equipado con diversas herramientas que
permiten al usuario \textbf{visualizar y modificar} su estado
de diversas maneras, tal y como se puede observar en la figura \ref{fig:mips-tools}.
Como detalle final, el simulador presenta una estructura basada en
\textbf{hilos}, lo que evita que \textit{JAMS} se congele al ejecutar
una aplicación.

\begin{figure}[!t]
    \centering
    \includegraphics[width=\textwidth]{images/result/mips-tools}
    \caption{Todas las herramientas proporcionadas por el simulador}
    \label{fig:mips-tools}
    \vspace{12cm}
\end{figure}

    \clearpage{\pagestyle{empty}\cleardoublepage}

    \chapter{Conclusiones}\label{ch:conclusiones}

Los objetivos de este proyecto están asociados a la \textbf{creación
de un nuevo entorno de desarrollo} especializado en lenguajes
ensamblador que puedan usar tanto desarrolladores avanzados
como estudiantes matriculados en los grados de titulaciones
relacionadas con la informática.

\textit{JAMS} es un entorno de desarrollo moderno,
flexible, modular y fácil de utilizar, donde el usuario puede
crear aplicaciones de una manera rápida y cómoda,
apoyándose en las diferentes herramientas y
características que aporta.

\textit{JAMS} se ha empaquetado junto con un
editor, un ensamblador y un simulador para la arquitectura
\textit{MIPS32}, permitiendo personalizar la manera en la
que se ejecuta un proyecto.

Gracias a la elección de tecnologías y
\textit{frameworks}, la experiencia proporcionada por
\textit{JAMS} es altamente personalizable, permitiendo
al usuario utilizar y producir temas e idiomas.


\section{Limitaciones}\label{sec:limitaciones}

\textit{JAMS} presenta las siguientes limitaciones:

\begin{itemize}
    \item \textbf{Velocidad}: \textit{JAMS} es capaz de ejecutar
    una simulación a un máximo de \textbf{40 millones de instrucciones por segundo} en un
    procesador \textit{AMD Ryzen 7 2700X}.
    Esta velocidad, aunque supere con creces a la mayoría de simuladores \textit{MIPS32},
    puede impedir ejecutar aplicaciones que requieran de una gran capacidad de cómputo.
    \item \textbf{Soporte para MIPS32r5}: actualmente, \textit{JAMS} solo da soporte
    a la última revisión de la arquitectura \textit{MIPS32}.
    Muchos usuarios siguen empleando la revisión anterior, por lo que tendrán que migrar
    sus proyectos antes de poder emplear \textit{JAMS}.
    \item \textbf{Falta de un sistema de componentes}: aunque \textit{JAMS}
    goce de una arquitectura totalmente personalizable,
    aún no tiene la opción de cargar componentes.
    \item \textbf{Editor de texto}: al no poseer \textit{JavaFX} de un editor
    de texto capaz de inyectar estilos a secciones sel texto se ha tido que
    utilizar una librería externa poco amigable con el desarrollador y difícil
    de depurar.
    Este motivo, juntado con la dificultad de hacer un editor de texto inteligente,
    ha hecho que el desarrollo del editor sufriera varias iteraciones y
    cuente todavía con varios fallos en casos extremos.
    Si se volviera a rehacer el proyecto se crearía una librería propia para
    el editor de texto.
    \item \textbf{Falta de arquitecturas}: \textit{JAMS} permite simular
    un proyecto en tres arquitecturas diferentes: uniciclo, multiciclo y segmentado.
    Al comienzo del proyecto estaba planeado el incluir muchas más arquitecturas,
    como son las arquitecturas con planificación dinámica de instrucciones.
    Estas características no han sido implementadas en el proyecto por
    falta de tiempo y por la gran dificultad que conlleva su desarrollo.
\end{itemize}

\section{Líneas futuras}\label{sec:líneas-futuras}

Hay varias propuestas que no se han abordado en este Trabajo
Fin de Grado que sería de gran interés poner el desarrollo:
\begin{itemize}
    \item \textbf{Sistema de componentes}: una de las características
    más interesantes sería la creación de un sistema de componentes
    que permita a desarrolladores externos modificar el comportamiento
    de \textit{JAMS} añadiendo nuevas características y soporte para
    más tecnologías.
    \item \textbf{Soporte para la arquitectura \textit{MIPS32r5}}:
    actualmente, \textit{JAMS} solo soporta proyectos de la revisión 6
    de la arquitectura \textit{MIPS32}.
    Esta revisión cambia la arquitectura en muchos aspectos con respecto
    a la revisión anterior, lo que fuerza a muchas personas a tener
    que migrar gran parte del código de sus proyectos.
    Añadir soporte a la revisión 5 será una tarea relativamente sencilla,
    sabiendo que la revisión 6 añade más características de las que quita.
    \item \textbf{Uso del \textit{Proyecto Valhalla}:} el
    \textit{Proyecto Valhalla}\cite{PROJECT_VALHALLA}, creado en 2014,
    está desarrollando una de las características más esperadas
    por los desarrolladores \textit{JAVA}: \textbf{paso por valor}.
    Esta característica permitirá optimizar de manera considerable
    muchos aspectos de \textit{JAMS}, como son el editor o el simulador.
    Estas nuevas características empezarán su desarrollo cuando esta
    tecnología esté en fase \textit{preview}.
\end{itemize}


\section{Reflexiones finales}\label{sec:reflexiones-finales}

Este ha sido el proyecto más complejo en el que he trabajo:
\textit{JAMS} abarca un montón de conceptos y tecnologías,
desde la simulación de arquitecturas hasta el despliegue de
aplicaciones automatizado, pasando por el renderizado a tiempo
real, la creación de interfaces y la estructuración de un
proyecto grande.

\textit{JAMS} ha sido una apuesta, un proyecto
que podría haberse derrumbado rápidamente.
Puede considerarse la consolidación de todos los conocimientos
que he adquirido en los últimos años, tanto fuera como dentro
de la universidad.

Finalmente, agradecer a todas las personas que me han
estado apoyando en este proyecto, a mi familia, amigos y a mis
tutores Óscar y Luis, ya que gracias a ellos este trabajo
ha sido posible.

    \clearpage{\pagestyle{empty}\cleardoublepage}

    \backmatter
    \bibliography{main}
    \bibliographystyle{orobles}
    \addcontentsline{toc}{chapter}{Bibliografía}
\end{document}
