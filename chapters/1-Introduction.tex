\chapter{Descripción del problema} \label{ch:descripcion-del-problema}


\section{Definición de conceptos}\label{sec:definicion-de-conceptos}

\subsection{La arquitectura \textit{MIPS}}\label{subsec:la-arquitectura-mips}

Se conoce con el nombre de \textbf{procesadores \textit{MIPS}}
\footnote{Creado por \emph{MIPS Computer Systems}, en la actualidad \emph{MIPS Technologies}}
a una familia de microprocesadores que implementan
la arquitectura \textit{RISC} del mismo nombre.
Aunque esta arquitectura no ha tenido un gran éxito en los computadores personales, sí ha gozado de fama en
sistemas empotrados, enrutadores, videoconsolas como la \textit{Nintendo 64} o la \textit{PlayStation 2},
y en máquinas utilizadas en diferentes aplicaciones, como las estaciones \textit{SGI IRIS 4D}
creadas por \textit{Silicon Graphics, Inc.} y empleadas en la industria cinematográfica para
la creación de animaciones y la generación de efectos especiales.

Su simple diseño y baja curva de aprendizaje hace a esta arquitectura ideal para introducir a los alumnos
a los lenguajes ensambladores.

\subsection{\textit{IDEs}: entornos de desarrollo integrados}\label{subsec:ides-entornos-de-desarrollo-integrados}

Un \textbf{entorno de desarrollo integrado} (\textit{IDE, Integrated Development Environment}) es una aplicación
informática con diversas herramientas empleadas en el desarrollo, construcción y depuración de \textit{software}.
Un \textit{IDE} suele estar desarrollado con dos objetivos clave en mente: maximizar la productividad del
desarrollador y evitar que este tenga que utilizar herramientas externas en su trabajo.

Existe una gran variedad de entornos de desarrollo en el mercado.
Estos se pueden clasificar de diferentes maneras.

\textbf{Según su propósito:}
\begin{itemize}
    \item \textbf{Entornos de desarrollo específicos:} son entornos pensados para el desarrollo en una
    tecnología concreta.
    Algunos ejemplos son \textit{NetBeans}, \textit{CLion} o \textit{Android Studio}.
    \item \textbf{Entornos de desarrollo generales:} son entornos que no se centran en una tecnología en concreto.
    Algunos ejemplos son \textit{IntelliJ IDEA}\cite{INTELLIJIDEA},
    \textit{Eclipse}\cite{ECLIPSE} o \textit{Microsoft Visual Studio}\cite{VISUALSTUDIO}
\end{itemize}

\textbf{Según su complejidad:}
\begin{itemize}
    \item \textbf{Entornos de desarrollo complejos:} son aplicaciones pesadas con una gran cantidad de
    características y funcionalidades.
    Algunos ejemplos son \textit{IntelliJ IDEA},
    \textit{Microsoft Visual Studio} o \textit{Eclipse}.
    \item \textbf{Entornos de desarrollo ligeros:} son aplicaciones que ocupan muy poco espacio, donde
    las características más avanzadas suelen ser componentes descargables.
    Algunos ejemplos son \textit{Visual Studio Code}\cite{VISUALSTUDIOCODE},
    \textit{Atom}\cite{ATOM} o \textit{Notepad++}\cite{NOTEPADPP}.
\end{itemize}

\subsection{Simuladores y emuladores}
\label{subsec:simuladores-y-emuladores}

Tanto un simulador como un emulador puede definirse como un
programa de computador que imita el funcionamiento de uno o varios
componentes \textit{hardware}.
Un emulador o simulador imita a una arquitectura, permitiendo
ejecutar aplicaciones ajenas a la arquitectura del computador anfitrión.

Es importante diferenciar los conceptos de \textbf{simulación}
y \textbf{emulación} a la hora de crear un programa que ejecute
código de arquitecturas externas.
La diferencia entre los simuladores y emuladores se manifiesta
en la manera en la que se implementan.

Un emulador tiene como objetivo \textbf{imitar el resultado}
que la arquitectura imitada produce al ejecutar un programa, sin tener
en cuenta el proceso interno que produce dicho resultado.
Los emuladores tienden a ser rápidos, intentando producir el resultado
de la manera más rápida y fiel posible.

Los simuladores tienen como objetivo \textbf{imitar el proceso}
que produce el resultado, simulando todos los componentes de la arquitectura.
Los simuladores tienden a ser más lentos que los emuladores, pero son más
adecuados para desarrollar y depurar aplicaciones para la arquitectura
imitada aunque no se disponga físicamente del \textit{hardware}.


\section{Descripción del problema}\label{sec:descripcion-del-problema}

Como se verá más adelante, los simuladores de la arquitectura \textit{MIPS} tuvieron
un auge importante a principios del milenio, con herramientas como \textit{MARS}\cite{MARS}
o \textit{EduMIPS64}\cite{EDUMIPS64}.
Estas herramientas estaban centradas principalmente en el \textbf{ámbito educativo}, proporcionando
interfaces sencillas y pocas herramientas.
Con el paso de los años, estas herramientas han quedado obsoletas, impidiendo desarrollar aplicaciones
en ensamblador \textit{MIPS32} de una manera cómoda.

Los principales problemas que presentan estas aplicaciones son los siguientes:
\begin{itemize}
    \item \textbf{Falta de herramientas}: los entornos de desarrollo \textit{MIPS} sufren de una
    carencia grave de herramientas importantes.
    Este problema afecta principalmente a programas como \textit{EduMIPS64}, donde no existe un editor de texto y
    el ensamblador es una aplicación separada que require la línea de comandos para funcionar.
    \item \textbf{Falta de estructura de proyecto}: a excepción de MARS, que lo hace de forma muy rudimentaria,
    ninguna aplicación actual presenta una estructura
    de proyecto, indispensable para el desarrollo de aplicaciones medianas y grandes.
    \item \textbf{Falta de personalización}: al ser aplicaciones muy sencillas, ninguna permite
    personalizar la apariencia de la interfaz.
    \item \textbf{Obsolescencia}: las aplicaciones están desarrolladas con tecnologías antiguas,
    con interfaces similares a las de los programas de principio de siglo.
    \item \textbf{Falta de capacidad de expansión}: el problema más grave que presentan estas aplicaciones
    es la poca capacidad de expansión de sus características, impidiendo que otros desarrolladores
    añadan nuevas funcionalidades de manera sencilla.
    \textit{MARS} es de los pocos entornos de desarrollo que permite expandir
    sus características mediante componentes,
    aunque su funcionamiento es muy rudimentario: el usuario debe modificar el archivo ejecutable
    para añadir nuevas herramientas.
\end{itemize}


\section{Objetivos}\label{sec:objetivos}

El objetivo principal de este proyecto es el desarrollo de \textbf{\textit{JAMS}
(\textit{Just Another MIPS Simulator})}, un \textbf{entorno de desarrollo integrado ligero, expandible y moderno}
centrado en los lenguajes ensambladores y, más concretamente,
en el lenguaje ensamblador de la arquitectura \textit{MIPS}.
Específicamente, en este apartado se detallarán diferentes objetivos.

\subsection{Investigar sobre las tecnologías más adecuadas para el desarrollo}
\label{subsec:investigar-sobre-las-tecnologias-mas-adecuados-para-el-desarrollo}

La tecnología usada en un proyecto tiene mucho peso en el resultado final.
Por ello, se debe buscar un conjunto de tecnologías que permitan crear una aplicación
moderna, multiplataforma y expandible.

Este objetivo puede separarse en dos pasos:
\begin{itemize}
    \item Elegir un lenguaje de programación con capacidad para crear aplicaciones modernas,
    que permita cargar código externo a voluntad y que tenga un buen rendimiento
    tanto en la ejecución como en el desarrollo.
    \item Seleccionar un \textit{framework} de desarrollo de aplicaciones de escritorio
    que permita desarrollar aplicaciones modernas y de gran
    calidad, dentro de las posibilidades del lenguaje de
    programación seleccionado.
\end{itemize}

\subsection{Crear un entorno base y un \textit{framework} que permita implementar diferentes herramientas}
\label{subsec:crear-un-entorno-base-y-un-framework-que-permita-implementar-diferentes-herramientas}

\textit{JAMS} pretende ser un entorno de desarrollo completamente modificable.
La mejor manera para conseguirlo es crear una base genérica
que se pueda ajustar posteriormente al desarrollo de una tecnología en concreto.

Esta base debe poder personalizar cualquier aspecto de la aplicación,
desde modificar simples mensajes y parámetros hasta añadir nuevas herramientas.
Estas modificaciones pueden venir de diferentes fuentes como
los paquetes de idiomas, los paquetes de temas y los complementos.

Por último, \textit{JAMS} debe permitir al usuario configurar los parámetros
de la aplicación de manera sencilla, proporcionando una interfaz de configuración
cómoda de utilizar.

Para conseguir este resultado, se proponen los siguientes subobjetivos:
\begin{itemize}
    \item Desarrollar una aplicación base en conjunto con una \textit{API} que pueda
    utilizarse por las diferentes herramientas para tecnologías concretas.
    \item Desarrollar e investigar diferentes formatos que permitan guardar
    y modificar los diferentes elementos estáticos de la aplicación, como
    los idiomas, los temas y la configuración, por ejemplo.
\end{itemize}

Este objetivo está relacionado con los
objetivos del Trabajo Fin de Grado del Grado en
Diseño y Desarrollo de Videojuegos de
la misma autoría que este Trabajo Fin de Grado:
desarrollar un \textbf{sistema de carga
dinámica de componentes} que permita al usuario activar y desactivar
componentes a voluntad sin tener que reiniciar la aplicación, así como
desarrollar tecnologías que permitan \textbf{modificar}
todos los aspectos de \textit{JAMS} mediante componentes.

\subsection{Diseñar e implementar un entorno de desarrollo para la arquitectura \textit{MIPS32}}
\label{subsec:desarrollar-un-entorno-de-desarrollo-para-la-arquitectura-mips32}

Una vez se tenga una base sólida se desarollará un editor, un ensamblador
y un simulador para la arquitectura \textit{MIPS32r6}.
Estas tres tecnologías se apoyarán en diferentes herramientas que complementarán
su utilización, como puede ser el caso del explorador en el editor y el visualizador
de memoria en el simulador.

Este objetivo se desarrollará en dos pasos:

\begin{itemize}
    \item Desarrollar un editor, un ensamblador y un simulador usando la tecnología
    proporcionada por la base.
    Estos tres componentes deben ser expandibles mediante componentes externos.
    \item Desarrollar diferentes herramientas que complementen las funcionalidades
    del editor, del ensamblador y del simulador.
\end{itemize}

Cabe destacar que no es un objetivo el permitir crear código válido para
entornos \textit{MIPS} reales.


\section{Metodología}\label{sec:metodologia}

Debido a que la aplicación está dividida en tres secciones (tecnologías, base y entorno de desarrollo \textit{MIPS}),
es muy importante marcar una metodología que permita un desarrollo consistente, eficiente y rápido.
A nivel de desarrollo se ha optado por una metodología ágil, con \textit{sprints} de varios meses
donde se desarrollan una serie de características claves.
Todas las características se someten a varias iteraciones donde se modifican y mejoran hasta lograr un
resultado idóneo.

También se ha seguido un desarrollo basado en pruebas unitarias,
permitiendo mantener la calidad del código mientras la aplicación va evolucionando.

Toda la metodología ha sido implementada mediante las herramientas proporcionadas por \textit{GitHub}.
Los estados de las características asignadas a un \textit{sprint} son documentados mediante proyectos,
como se observa en la figura \ref{fig:introduccion-github}.
Las acciones obligan a que las pruebas unitarias deban superarse sin errores si se desea añadir una nueva
característica a la rama principal.
Estas acciones también se emplean para generar los binarios cuando se supera un \textit{sprint}
y se lanza una nueva versión de \textit{JAMS}.

\begin{figure}[H]
    \centering
    \includegraphics[width=\textwidth]{images/introduction/github}
    \caption{Proyecto de \textit{GitHub} para la versión 0.4-BETA}
    \label{fig:introduccion-github}
\end{figure}
