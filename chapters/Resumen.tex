\chapter{Resumen} \label{chapter:resumen}
Este Trabajo de Fin de Máster ha sido desarrollado dentro del Grupo de Modelado y Realidad Virtual (GMRV) de la Universidad Rey Juan Carlos de Madrid (UJRC). Se enmarca dentro del ámbito del análisis de datos de simulación neurocientífica. Por ello, este Trabajo contempla el diseño e implementación de dos plataformas, como una posible solución al tratamiento de datos de simulación neurocientífica (aunque extendible a otros ámbitos).\\
En estos últimos años el estudio del cerebro ha sido uno de los temas más estudiadas gracias a diversos proyectos internacionales como el Human Brain Project, Brain o Blue Brain Project, además del gran uso de las redes neuronales artificiales para mejorar los algoritmos informáticos.\\
Sin duda, este tipo de redes son altamente complejas y pueden llegar a producir efectos inesperados, por lo que es necesario realizar una análisis de los datos obtenidos para comprobar si la simulación es correcta.\\
Por esta razón, \textit{TEVimos} y \textit{CuViz} buscan posibles soluciones gráficas para análisis de datos de simulación a través de mapas de elevación y de resúmenes de simulación, respectivamente.\\
Este proyecto provee dos tipos de representación muy diferenciados: Simulación en tiempo real gracias a \textit{TEVimos} y resumen de simulación (offline) a través de \textit{CuViz}.\\
La finalidad de ambas plataformas es conseguir que cualquier centro de trabajo pueda visualizar los datos de sus simulaciones para comprobar, por ejemplo, si se han realizado correctamente o detectar patrones de actividad en las neuronas. Con esto, se podrá contribuir en el avance en el conocimiento del funcionamiento de las neuronas y el cerebro en su conjunto final.