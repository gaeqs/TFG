\begin{itemize}
    \item \textbf{WinMIPS64}\cite{WINMIPS64}: simulador
    de la arquitectura \textit{MIPS64} que implementa
    un procesador segmentado, donde las instrucciones
    pueden tardar varios ciclos en ejecutarse.
    La herramienta más interesante de la aplicación
    es su \textbf{visualizador de flujo}.
    Existe un clon llamado \textit{EduMIPS64} desarrollado
    en \textit{Java} y multiplataforma.
    \item \textbf{Simula3MS}\cite{SIMULA3MS}: desarrollado en
    la Universidad de A Coruña, \textit{Simula3MS} es un
    pequeño entorno de desarrollo para \textit{MIPS32}
    con un simulador que permite \textbf{ejecutar el código en
    diferentes tipos de procesadores}.
    Está desarrollado en \textit{Java} con la librería \textit{Swing}.
    \item \textbf{DrMIPS}\cite{DRMIPS}: presenta un
    editor de texto, un ensamblador y un simulador
    para \textit{MIPS32}.
    La aplicación tiene un enfoque educativo, teniendo el simulador
    un visualizador de los caminos del procesador.
    Esto permite
\end{itemize}