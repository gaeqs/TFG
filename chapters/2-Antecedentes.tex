\chapter{Antecedentes}\label{ch:antecedentes}


\section{Antecedentes en los entornos de desarrollo \textit{MIPS}}
\label{sec:antecedentes-en-los-entornos-de-desarrollo-mips}

Los entornos de desarrollo \textit{MIPS} han estado presentes en
el mercado desde el principio de los años 90, quedando su desarrollo
estancado en la década de\delODR{l} 2010.
Estos entornos de desarrollo suelen estar enfocados al ámbito
educativo, con implementaciones muy básicas de la arquitectura.

A continuación se presentarán los principales
entornos de desarrollo y simuladores \textit{MIPS}:
\begin{itemize}
    \item \textbf{SPIM}\cite{SPIM}: es uno de los primeros simuladores
    \textit{MIPS32}, y uno de los más influyentes.
    La aplicación cuenta con un ensamblador que soporta las
    instrucciones más básicas de las primeras revisiones de la arquitectura.
    Es una aplicación que \textbf{se controla principalmente por consola},
    aunque las últimas versiones incorporan una interfaz gráfica
    desarrollada con el \textit{framework Qt}\cite{QT}.
    Los simuladores posteriores \textbf{tomarían mucha inspiración} de \textit{SPIM},
    implementando las llamadas al sistema desarrolladas específicamente
    para este simulador.
    \item \textbf{MARS}\cite{MARS}: desarrollado en la Universidad
    del estado de Missouri en 2005, \textit{MARS} es uno de los
    entornos de desarrollo \textit{MIPS32} más populares en el ámbito educativo.
    La aplicación, desarrollada en \textit{Java}\cite{JAVA}
    usando la librería \textit{Swing}\cite{SWING},
    presenta un editor de texto básico con \textbf{autocompletador}, un
    ensamblador con soporte para una gran variedad de instrucciones y
    pseudo-instrucciones y un simulador básico con diferentes herramientas
    desplegables.
    \item \textbf{WinMIPS64}\cite{WINMIPS64}: simulador
    de la arquitectura \textit{MIPS64} que implementa
    un procesador segmentado donde las instrucciones
    pueden tardar varios ciclos en ejecutarse.
    La herramienta más interesante de la aplicación
    es su \textbf{visualizador de flujo}.
    Existe un clon llamado \textit{EduMIPS64} desarrollado
    en \textit{Java} y multiplataforma.
    \item \textbf{Simula3MS}\cite{SIMULA3MS}: desarrollado en
    la Universidad de A Coruña, \textit{Simula3MS} es un
    pequeño entorno de desarrollo para \textit{MIPS32}
    con un simulador que permite \textbf{ejecutar el código en
    diferentes tipos de procesadores}.
    Está desarrollado en \textit{Java} con la librería \textit{Swing}.
    \item \textbf{DrMIPS}\cite{DRMIPS}: presenta un
    editor de texto, un ensamblador y un simulador
    para \textit{MIPS32}.
    La aplicación tiene un enfoque educativo, teniendo el simulador
    un \textbf{visualizador de los caminos del procesador}.
    \item \textbf{VisualMIPS32}\cite{VISUALMIPS32}: entorno
    de desarrollo \textit{MIPS32} desarrollado y
    utilizado por la Universidad de Sevilla para la enseñanza.
    Cuenta con un editor, un ensamblador y un simulador.
    Igual que \textit{WinMIPS64}, cuenta con un visualizador
    de flujo integrado.
\end{itemize}


\section{Antecedentes en los entornos de desarrollo}
\label{sec:antecedentes-en-los-entornos-de-desarrollo}

\textit{JAMS} también está fuertemente inspirado en los
entornos de desarrollo de carácter general más modernos.
Estas aplicaciones presentan una serie de funcionalidades que
caracterizan a los \textit{IDEs} de hoy en día:

\begin{itemize}
    \item \textbf{Nodos}: también conocidos como \textit{Widgets} o \textit{Tools}.
    Consisten en pequeñas herramientas que complementan al entorno.
    Antiguamente, estas herramientas solían desplegarse en una ventana diferente,
    siendo necesario seleccionarlas desde un menú desplegable en la ventana principal.
    Actualmente, los nodos suelen estar integrados \newODR{a} los lados de la ventana principal,
    siendo estos capaces de ser desplegados pulsando un botón en las barras laterales
    de la aplicación.
    Estos nodos son altamente personalizables, pudiendo el usuario cambiar su posición
    y el modo de despliegue.
    \item \textbf{Proyectos}: los entornos de desarrollo modernos están estructurados
    alrededor del concepto de proyecto.
    Un proyecto es un conjunto de archivos que componen el código fuente de una aplicación.
    Actualmente, no existe ningún entorno de desarrollo especializado en \textit{MIPS32}
    que presente esta característica.
\end{itemize}

Actualmente, existe una gran cantidad de entornos de desarrollo en el mercado:
específicos y generales, complejos y ligeros.
Cada uno de ellos presenta unas características diferentes que los hace único.

A continuación se presentarán algunos de los principales
entornos de desarrollo:
\begin{itemize}
    \item \textbf{Microsoft Visual Studio}\cite{VISUALSTUDIO}: entorno de desarrollo
    integrado general y complejo desarrollado por \textit{Microsoft}.
    Está programado en C++ y C\#, con soporte para componentes.
    Estos componentes son los que permiten añadir soporte para diversas tecnologías
    al entorno de desarrollo, ya que de por sí tiene soporte para una pequeña
    variedad de tecnologías.
    Junto a \textit{Intellij IDEA}, \textit{Microsoft Visual Studio} ha sido uno
    de los pioneros en integrar sus herramientas a la ventana principal.
    \textit{Microsoft Visual Studio} está disponible en tres versiones: la versión
    \textit{Community} para uso no comercial, la versión \textit{Professional} para
    proyectos comerciales y la versión \textit{Enterprise}, siendo esta la versión
    con más características.
    Las tres versiones son de \textbf{código privativo}.
    \item \textbf{Visual Studio Code}\cite{VISUALSTUDIOCODE}:
    es la alternativa de código abierto de \textit{Microsoft}.
    \textit{Visual Studio Code} es un entorno de desarrollo general
    y ligero \textit{desarrollado en HTML/CSS/TS} que sigue la misma filosofía
    de componentes de \textit{Microsoft Visual Studio}.
    \item \textbf{Eclipse}\cite{ECLIPSE}: creado por \textit{IBM} y actualmente
    desarrollado por \textit{Eclipse Foundation}.
    \textit{Eclipse} es entorno de desarrollo complejo desarrollado en \textit{Java},
    centrándose principalmente en el \textbf{desarrollo de aplicaciones para dicho lenguaje}.
    Actualmente, \textit{Eclipse} se ha convertido en un \textit{IDE} general que
    permite desarrollar en más de 20 lenguajes de programación.
    \textit{Eclipse} es de código abierto y totalmente gratis.
    \item \textbf{Intellij IDEA / \textit{IDEs} de \textit{JetBrains}}\cite{INTELLIJIDEA}:
    \textit{JetBrains} sigue una filosofía diferente al introducir
    soporte a un nuevo lenguaje de programación.
    Al principio, lanzan un componente para todos sus entornos de desarrollo,
    el cual da la habilidad de desarrollar en una nueva tecnología.
    Una vez el componente es lo suficientemente maduro, \textit{JetBrains}
    \textbf{lanza un nuevo entorno de desarrollo} para el lenguaje de programación
    en concreto.
    Todos estos entornos de desarrollo tienen la misma base, creada a
    principios de los años 2000 para \textit{Intellij IDEA}, su primer \textit{IDE}.
    Salvo algunas excepciones, todos los entornos de desarollo de \textit{JetBrains}
    son de código privativo y de pago, necesitando una licencia para poder usarlos.
    Las únicas excepciones son las versiones \textit{Community} de \textit{Intellij IDEA}
    y \textit{PyCharm}, de código abierto y para uso no comercial.
\end{itemize}
