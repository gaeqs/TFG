\chapter{Conclusiones}\label{ch:conclusiones}

Los objetivos de este proyecto están asociados a la \textbf{creación
de un nuevo entorno de desarrollo} especializado en lenguajes
ensamblador que puedan usar tanto desarrolladores avanzados
como estudiantes matriculados en los grados de titulaciones
relacionadas con la informática.

\textit{JAMS} es un entorno de desarrollo moderno,
flexible, modular y fácil de utilizar, donde el usuario puede
crear aplicaciones de una manera rápida y cómoda,
apoyándose en las diferentes herramientas y
características que aporta.

\textit{JAMS} se ha empaquetado junto con un
editor, un ensamblador y un simulador para la arquitectura
\textit{MIPS32}, permitiendo personalizar la manera en la
que se ejecuta un proyecto.

Gracias a la elección de tecnologías y
\textit{frameworks}, la experiencia proporcionada por
\textit{JAMS} es altamente personalizable, permitiendo
al usuario utilizar y producir temas e idiomas.


\section{Limitaciones}\label{sec:limitaciones}

\textit{JAMS} presenta las siguientes limitaciones:

\begin{itemize}
    \item \textbf{Velocidad}: \textit{JAMS} es capaz de ejecutar
    una simulación a un máximo de \textbf{40 millones de instrucciones por segundo} en un
    procesador \textit{AMD Ryzen 7 2700X}.
    Esta velocidad, aunque supere con creces a la mayoría de simuladores \textit{MIPS32},
    puede impedir ejecutar aplicaciones que requieran de una gran capacidad de cómputo.
    \item \textbf{Soporte para MIPS32r5}: actualmente, \textit{JAMS} solo da soporte
    a la última revisión de la arquitectura \textit{MIPS32}.
    Muchos usuarios siguen empleando la revisión anterior, por lo que tendrán que migrar
    sus proyectos antes de poder emplear \textit{JAMS}.
    \item \textbf{Falta de un sistema de componentes}: aunque \textit{JAMS}
    goce de una arquitectura totalmente personalizable,
    aún no tiene la opción de cargar componentes.
    \item \textbf{Editor de texto}: al no poseer \textit{JavaFX} de un editor
    de texto capaz de inyectar estilos a secciones sel texto se ha tido que
    utilizar una librería externa poco amigable con el desarrollador y difícil
    de depurar.
    Este motivo, juntado con la dificultad de hacer un editor de texto inteligente,
    ha hecho que el desarrollo del editor sufriera varias iteraciones y
    cuente todavía con varios fallos en casos extremos.
    Si se volviera a rehacer el proyecto se crearía una librería propia para
    el editor de texto.
    \item \textbf{Falta de arquitecturas}: \textit{JAMS} permite simular
    un proyecto en tres arquitecturas diferentes: uniciclo, multiciclo y segmentado.
    Al comienzo del proyecto estaba planeado el incluir muchas más arquitecturas,
    como son las arquitecturas con planificación dinámica de instrucciones.
    Estas características no han sido implementadas en el proyecto por
    falta de tiempo y por la gran dificultad que conlleva su desarrollo.
\end{itemize}

\section{Líneas futuras}\label{sec:líneas-futuras}

Hay varias propuestas que no se han abordado en este Trabajo
Fin de Grado que sería de gran interés poner el desarrollo:
\begin{itemize}
    \item \textbf{Sistema de componentes}: una de las características
    más interesantes sería la creación de un sistema de componentes
    que permita a desarrolladores externos modificar el comportamiento
    de \textit{JAMS} añadiendo nuevas características y soporte para
    más tecnologías.
    \item \textbf{Soporte para la arquitectura \textit{MIPS32r5}}:
    actualmente, \textit{JAMS} solo soporta proyectos de la revisión 6
    de la arquitectura \textit{MIPS32}.
    Esta revisión cambia la arquitectura en muchos aspectos con respecto
    a la revisión anterior, lo que fuerza a muchas personas a tener
    que migrar gran parte del código de sus proyectos.
    Añadir soporte a la revisión 5 será una tarea relativamente sencilla,
    sabiendo que la revisión 6 añade más características de las que quita.
    \item \textbf{Uso del \textit{Proyecto Valhalla}:} el
    \textit{Proyecto Valhalla}\cite{PROJECT_VALHALLA}, creado en 2014,
    está desarrollando una de las características más esperadas
    por los desarrolladores \textit{JAVA}: \textbf{paso por valor}.
    Esta característica permitirá optimizar de manera considerable
    muchos aspectos de \textit{JAMS}, como son el editor o el simulador.
    Estas nuevas características empezarán su desarrollo cuando esta
    tecnología esté en fase \textit{preview}.
\end{itemize}


\section{Reflexiones finales}\label{sec:reflexiones-finales}

Este ha sido el proyecto más complejo en el que he trabajo:
\textit{JAMS} abarca un montón de conceptos y tecnologías,
desde la simulación de arquitecturas hasta el despliegue de
aplicaciones automatizado, pasando por el renderizado a tiempo
real, la creación de interfaces y la estructuración de un
proyecto grande.

\textit{JAMS} ha sido una apuesta, un proyecto
que podría haberse derrumbado rápidamente.
Puede considerarse la consolidación de todos los conocimientos
que he adquirido en los últimos años, tanto fuera como dentro
de la universidad.

Finalmente, agradecer a todas las personas que me han
estado apoyando en este proyecto, a mi familia, amigos y a mis
tutores Óscar y Luis, ya que gracias a ellos este trabajo
ha sido posible.
