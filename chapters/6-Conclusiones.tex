\chapter{Conclusiones}\label{ch:conclusiones}

Los objetivos de este proyecto estaban asociados a la \textbf{creación
de un nuevo entorno de desarrollo} especializado en lenguajes
ensamblador que pudieran usar tanto desarrolladores avanzados
como alumnos.

\noindent \textit{JAMS} es un entorno de desarrollo moderno,
flexible, modular y fácil de usar, donde el usuario puede
crear aplicaciones de una manera rápida y cómoda,
apoyándose en las diferentes herramientas y
características que la aplicación aporta.

\noindent La aplicación viene empaquetada junto con un
editor, un ensamblador y un simulador para la arquitectura
\textit{MIPS32}, permitiendo personalizar la manera en la
que un proyecto es ejecutado.

\noindent Gracias a la buena elección de tecnologías y
\textit{frameworks}, la experiencia proporcionada por
\textit{JAMS} es altamente personalizable, permitiendo
al usuario usar y crear temas e idiomas.

\noindent Aunque \textit{JAMS} goce de una arquitectura
totalmente modular, aún no tiene la opción de cargar componentes.
Esta característica será implementada en el Trabajo de Fin de
Grado del Grado en Diseño y Desarrollo de Videojuegos.

\section{Líneas futuras}\label{sec:líneas-futuras}

\textit{JAMS} contará con dos actualizaciones importantes en el
futuro cercano.
La primera actualización ya está en desarrollo, mientras que la
segunda actualización requerirá de nueva tecnología que
el equipo de \textit{JAVA} está desarrollando.
\begin{itemize}
    \item \textbf{Soporte para la arquitectura \textit{MIPS32r5}}:
    actualmente, \textit{JAMS} solo soporta proyectos de la revisión 6
    de la arquitectura \textit{MIPS32}.
    Esta revisión cambia la arquitectura en muchos aspectos con respecto
    a la revisión anterior, lo que fuerza a muchas personas a tener
    que migrar gran parte del código de sus proyectos.
    Añadir soporte a la revisión 5 será una tarea relativamente sencilla,
    sabiendo que la revisión 6 añade más características de las que quita.
    \item \textbf{Uso del \textit{Proyecto Valhalla}:} el
    \textit{Proyecto Valhalla}\cite{PROJECT_VALHALLA}, creado en 2014,
    está desarrollando una de las características más esperadas
    por los desarrolladores \textit{JAVA}: \textbf{paso por valor}.
    Esta característica permitirá optimizar de manera considerable
    muchos aspectos de \textit{JAMS}, como son el editor o el simulador.
    Estas nuevas características empezarán su desarrollo cuando esta
    tecnología esté en fase \textit{preview}.
\end{itemize}

\section{Reflexiones finales}\label{sec:reflexiones-finales}

Este ha sido el proyecto más complejo en el que he trabajo:
\textit{JAMS} abarca un montón de conceptos y tecnologías,
desde la simulación de arquitecturas hasta el despliegue de
aplicaciones automatizado, pasando por el renderizado a tiempo
real, la creación de interfaces y la estructuración de un
proyecto grande.

\noindent \textit{JAMS} ha sido una apuesta, un proyecto
que podría haberse derrumbado rápidamente.
Puede considerarse la consolidación de todos los conocimientos
que he adquirido en los últimos años, tanto fuera como dentro
de la universidad.

\noindent Finalmente, agradecer a todas las personas que me han
estado apoyando en este proyecto, a mi familia, amigos y a mis
tutores Óscar y Luis, ya que gracias a ellos este trabajo
ha sido posible.